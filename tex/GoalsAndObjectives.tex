\begin{frame}{Darbo tikslai ir uždaviniai}
    \textbf{Darbo tikslas}. Atlikti VQLS-SVM algoritmo tyrimą, siekiant išplėsti jo taikymą iki 10 kubitų.

    \textbf{Darbo uždaviniai.}
    \begin{itemize}
         \item Išanalizuoti VQLS-SVM algoritmą \cite{VQLS-SVM} straipsnyje, pritaikyti kodo optimizavimus, skirtus sumažinti algoritmo skaičiavimo trukmę.
    \item Atlikti tyrimus su skirtingomis duomenų aibėmis, palyginti klasikinio ir kvantinio atraminių vektorių klasifikatorių tikslumus.
    \item Atlikti įvairių gradientinių ir negradientinių optimizatorių bandymus, rasti efektyviausią optimizatorių problemos sprendimui.
    \item Išplėsti \cite{VQLS-SVM} straipsnyje pristatytą algoritmą iki 10 kubitų, siekiant padidinti apmokymo duomenų aibės dydį ir duomenų požymių skaičių. Toks išplėtimas leistų algoritmui efektyviai apdoroti sudėtingesnius duomenis ir padidinti klasifikavimo tikslumą.
    \item Pritaikyti Luko Hantzko straipsnyje \cite{TensorPauliDecomposition} pristatytą efektyvų tenzorinės Paulio bazės dekompozicijos algoritmą, kai kubitų skaičius pasiekia 10. 
    \item Ištirti ar Andrea Mari pristatytame straipsnyje nuoseklus variacinis kvantinis tiesinių lygčių sistemų algoritmas (\emph{angl. coherent variational quantum linear solver}) turi tam tikrų privalumų sprendžiant atraminių vektorių klasifikatoriaus problemą lyginant su VQLS algoritmu.\cite{CVQLS, VQLS}.
    \end{itemize}
\end{frame}