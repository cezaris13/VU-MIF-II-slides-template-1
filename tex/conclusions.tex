\section{Išvados}

\begin{frame}{Išvados}
    % \begin{enumerate}
    %     \item VQLS-SVM algoritmas tiksliai klasifikuoja testavimo duomenų rinkinį, kai apmokymo duomenų aibė turi tik 7 įrašus. Kvantinio atraminių vektorių klasifikatoriaus algoritmo tikslumas yra šiek tiek mažesnis lyginant su klasikiniu algoritmu. Gautas vektorius $x$ nėra tikslus sprendinys, kas turi įtakos klasifikavimo tikslumui. 
    %     \item Atlikus VQLS-SVM algoritmo tyrimą, buvo pastebėta, kad požymių skaičiui esant didesniam negu apmokymo duomenų aibė, klasifikavimo tikslumas yra mažesnis lyginant su duomenų rinkiniu, kurio požymių skaičius mažesnis už apmokymo aibę. 
    %     \item \textit{COBYLA} optimizatorius, iš tirtų išvestinių nereikalaujančių optimizatorių, greičiausiai pasiekė nuostolių funkcijos minimalią reikšmę. \textit{SQLSP} optimizavimo funkcija gali būti efektyvi optimizavimo funkcijos alternatyva, kai reikalingas didesnis skaičiavimų stabilumas, nes naudojant šį algoritmą, nuostolių funkcijos reikšmė didėjant iteracijų skaičiui, nedidėjo.
    % \end{enumerate}
\end{frame}

% \begin{frame}
%     %\frametitle{A first slide}
%     \begin{center}
%         \Huge Thank you for your attention!
%     \end{center}
% \end{frame}
