\section{Benefits}

 \begin{frame}[c]{Ateities darbai}
        % \begin{enumerate}
        %     \item Išplėsti \textit{ansatz} grandinę iki 10 kubitų, siekiant padidinti apmokymo duomenų aibės dydį ir duomenų požymių skaičių. Toks išplėtimas leistų algoritmui efektyviai apdoroti sudėtingesnius duomenis ir padidinti klasifikavimo tikslumą.
        %     \item Didinant \textit{ansatz} grandinę iki 10 kubitų galime susidurti, kad matricos išskaidymas naudojantis  Qiskit pristatyta SparsePauliOP funkcija, truks ženkliai ilgiau, lyginant su $3$ kubitų atveju. Lukas Hantzko straipsnyje pristatytas efektyvus tenzorinės Paulio bazės dekompozicijos algoritmas, kai kubitų skaičius pasiekia 10 \cite{TensorPauliDecomposition}. 
    
        %     \item Pritaikyti gradientinio nusileidimo optimizavimo algoritmus (ADAM,SPSA) ir atlikti tyrimus.
        %     \item Ištirti ar Andrea Mari pristatytame straipsnyje nuoseklus variacinis kvantinis tiesinių lygčių sistemų algoritmas (\emph{angl. coherent variational quantum linear solver}) turi tam tikrų privalumų sprendžiant atraminių vektorių klasifikatoriaus problemą \cite{CVQLS}.
        % \end{enumerate} 
\end{frame}
